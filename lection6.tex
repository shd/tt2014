\section{О классе функций, определимых в просто типизированном лямбда-исчислении}

%\begin{definition}
%Будем 
%\end{definition}

\begin{definition}
Назовем \emph{высотой} типа $h(\tau)$ следующее выражение:
$$h(\tau) = \left\{\begin{array}{ll}0&\tau\equiv\alpha\\
\max(h(\rho),h(\sigma))+1 & \tau\equiv\rho\rightarrow\sigma\end{array}\right.$$
\end{definition}

\begin{definition}
Назовем расширенным полиномом функцию $E: \Bbb N_0\times \Bbb N_0 \rightarrow \Bbb N_0$ следующего вида:
$$E(x,y) = \left\{\begin{array}{ll}a_{0,0} + a_{1,0}x + a_{0,1}y + a_{1,1}xy + \dots + a_{m,n}x^my^n & x>0,y>0 \\
b_0 + b_1x + \dots + b_k & x > 0, y = 0 \\
c_0 + c_1y + \dots + c_l & x = 0, y > 0\\
k & x = 0, y = 0
\end{array}\right.$$
\end{definition}

\begin{definition}
Пусть тип $\nu$ --- это $(\alpha\rightarrow\alpha)\rightarrow(\alpha\rightarrow\alpha)$.
Пусть $n$ --- некоторое натуральное число.
Выражение $\overline{n} \equiv \lambda f^{\alpha\rightarrow\alpha} . \lambda x^\alpha. f^n x$ 
назовем чёрчевским нумералом, соответствующим числу $n$.
\end{definition}

\begin{theorem}
При фиксированном типе для целых чисел 
$\nu=(\alpha\rightarrow\alpha)\rightarrow(\alpha\rightarrow\alpha)$
в типизированном исчислении по Чёрчу класс двуместных функций ограничен 
расширенными полиномами. То есть, каков бы ни был замкнутый лямбда-терм $R$, 
такой что $\vdash_\texttt{ч} R: \nu \rightarrow \nu \rightarrow \nu$,
найдется расширенный полином $E(m,n)$, такой, что
$R \,\overline{m}\,\overline{n} =_\beta \overline{E(m,n)}$.
\end{theorem}

\begin{lemma}
Если в выражении $X^\xi$, находящемся в нормальной форме,
подтерм $T^\tau$ не является свободной переменной выражения
$T$, и $T \ne X$, то всегда найдется такой подтерм $S^\sigma$, что 
$h(\sigma) > h(\tau)$, причем $\sigma=\tau\rightarrow\rho$ или
$\sigma=\rho\rightarrow\tau$.
\end{lemma}
\begin{proof}
Рассмотрим подтерм $T$. Возможны следующие варианты:

\begin{enumerate}
\item $T$ --- это некоторая переменная $x$ (она обязана быть связанной по условию леммы).
То есть $T$ --- часть выражения $S = \lambda x:\tau. \dots x \dots$. 
Тогда $S: \tau\rightarrow\rho$, и $h(\tau\rightarrow\rho) > h(\tau)$.
\item $T$ --- это некоторая абстракция $T=\lambda x:\sigma.P^\pi$. 
Тогда заметим, что по условию $T \ne X$. Значит, $T$ входит в некоторое более
общее выражение --- либо в абстракцию
$S^{\upsilon\rightarrow\tau} = \lambda y:\upsilon.T$, либо
в применение $S^{\tau\rightarrow\upsilon} T$
(применение вида $T A$ является редексом и потому невозможно).
\item $T$ --- это некоторое применение $T=S^\upsilon\rightarrow\tau Y$.
\end{enumerate}

\end{proof}

\begin{lemma} 
%$(\lambda f.\lambda x.f^m x) (\lambda x.g^n x) \mbredmath (\lambda x.g^{m \cdot n} x)$
$(\lambda t.g^n t)^m x \mbredmath g^{m \cdot n} x$.
\end{lemma}
\begin{proof} Индукция по m:% покажем, что 
%$(\lambda t.g^n t)^m x \mbredmath g^{m \cdot n} x$.
\begin{itemize}
\item[База] Пусть $m=0$. Тогда $(\lambda t.g^n t)^0 x \equiv x \equiv g^{0 \cdot n} x$.
%$(\lambda f.\lambda x.x) (\lambda x.g^n x) \bredmath \lambda x.x$
\item[Переход] Пусть $(\lambda t.g^n t)^m x \mbredmath g^{m \cdot n} x$.
Тогда $$(\lambda t.g^n t)^{m+1} x \equiv (\lambda t.g^n t)^m ((\lambda t.g^n t) x) \mbredmath 
g^{m \cdot n} ((\lambda t.g^n t) x) \bredmath g^{m \cdot n} (g^n x) \equiv g^{(m+1) \cdot n} x$$
%$(\lambda f.\lambda x.f^m x) (\lambda x.g^n x) \mbredmath \lambda x.g^{m \cdot n} x$.
%Тогда $$(\lambda f.\lambda x.f^{m+1} x) (\lambda x.g^n x) \mbredmath \lambda x.(\lambda x.g^n x)^m ((\lambda x.g^n x) x)
%\mbredmath \lambda x.g^{m \cdot n} ((\lambda x.g^n x) x) \mbredmath \lambda x.g^{m \cdot n} (g^n x) \equiv 
%\lambda x.g^{(m+1) \cdot n} x$$.
\end{itemize}
\end{proof}

\begin{proof}[Доказательство теоремы]
Рассмотрим лямбда-терм $R \, a^\nu \, b^\nu \, f^{\alpha\rightarrow\alpha}$.
Очевидно, что $R: \alpha\rightarrow\alpha$.

Согласно свойству сильной нормализации, данный терм имеет нормальную форму $N$. 
Рассмотрим ее. Заметим, что если $T^\tau$ --- подтерм $N$, то он обязан иметь тип 
либо $\nu$, либо $\alpha\rightarrow\alpha$, либо $\alpha$.

%\begin{tabular}{ll}
%Тип & Возможный вид терма\\
%\hline
%$\nu$ & $a$, $b$\\
%$\alpha\rightarrow\alpha$ & $a f$, $b f$
%f, $a S_1^{\alpha\rightarrow\alpha}$, $b S_1^{\alpha\rightarrow\alpha}$,
%	 $\lambda x:\alpha.S_1^{\alpha\rightarrow\alpha} (S_2^{\alpha\rightarrow\alpha} \dots S_n^{\alpha\rightarrow\alpha} (y) \dots)$\\
%$\alpha$ & $f$, связанная переменная $y:\alpha$, $S_1^{\alpha\rightarrow\alpha} (T^\alpha)$
%\end{tabular}

Доказать это можно разбором случаев с использованием индукции и предыдущей леммы.
Заметим, что в выражении не может быть выражений атомарных типов, отличных от
$\alpha$ (поскольку у нас запрещены свободные переменные).
Возьмем некоторый терм $T^\tau$ и рассмотрим $h(\tau)$.

\begin{enumerate}
\item Если $h(\tau) \ge 3$, то $T=\overline{a}$ или $T=\overline{b}$. 
Пусть это не так, и существуют такие $P^\pi$, что $P\ne \overline{a}$, $P\ne \overline{b}$ и $h(\pi) \ge 3$.
Возьмем среди таких $P$ подтерм с типом максимальной глубины.
Однако, по лемме в нем неизбежно найдется такой $S^\sigma$, что $h(\sigma)>h(\pi)$, 
что противоречит максимальности $h(\pi)$.

\item Если $h(\tau) = 2$, то $\tau$ имеет вид либо $\alpha\rightarrow(\alpha\rightarrow\alpha)$,
либо $(\alpha\rightarrow\alpha)\rightarrow\alpha$. По лемме найдется такой
$S^\sigma$, что $\sigma=\tau\rightarrow\rho$ или $\sigma=\rho\rightarrow\tau$.
В любом из случаев не найдется такого $\rho$, что $\nu = \sigma$, то есть $S\ne a$ и
$S\ne b$, что невозможно по предыдущему пункту.

\item $h(\tau) = 0$ или $h(\tau) = 1$. Тогда очевидно, что
$\tau\equiv\alpha$ или $\tau\equiv\alpha\rightarrow\alpha$ соответственно.
\end{enumerate}

Теперь рассмотрим терм $S^{\alpha\rightarrow\alpha}$. 
%Введем обозначение $\overline{m}=\lambda f:\alpha\rightarrow\alpha.\lambda x:\alpha.f^m x$.
Рассмотрим, какие выражения могут иметь такой тип.
Можно показать, что это будет либо:
\begin{enumerate}
\item $f$
\item $a T^{\alpha\rightarrow\alpha}$ или $b T^{\alpha\rightarrow\alpha}$
\item $\lambda y.S_1 ( \dots S_k Z \dots )$, где $S_i$ --- это либо $f$, либо $a f$ или $b f$,
а $Z$ либо совпадает с $y$, либо является некоторой другой переменной из объемлющей лямбда-абстракции.
\end{enumerate}

Пусть $S$ --- подтерм $N$ типа $\alpha\rightarrow\alpha$.
Покажем по индукции по структуре $S$, что $S[a := \overline{m}, b := \overline{n}] =_\beta f^{\overline{E(m,n)}}$, 
либо $\lambda y.f^{\overline{E(m,n)}} z$ (для некоторых $m,n\in \mathbb{N}_0$).

Разберем случаи:

\begin{enumerate}
\item $S\equiv f$ --- тогда $E(m,n)=1$ и $S\equiv f^1$.
\item $S\equiv a T$ (случай $b T$ рассматривается аналогично) --- тогда:

Пусть $T\equiv f^{\overline{E(m,n)}}$, тогда
$$a[a := \overline{m}] T \equiv (\lambda fx.f^m x) (\lambda x.f^{E(m,n) x)}$$
По лемме это выражение бета-эквивалентно такому:
$$\lambda x.(f^{E(m,n)})^m x \equiv \lambda x.f^E_1(m,n) x$$

Аналогично, если $T\equiv \lambda y.f^{\overline{E(a,b)}} z$, то
$a[a := \overline{m}] T \equiv (\lambda fx.f^{\overline{m}} x) \lambda y.f^{\overline{E(m,n)}} z 
=_\beta \lambda y.f^{\overline{E(m,n)}}z$
\end{enumerate}

%Т.о., в выражении возможны 3 только типа элементарных формул, типов $\alpha$, $\alpha\rightarrow\alpha$ и
%$(\alpha\rightarrow\alpha)\rightarrow(\alpha\rightarrow\alpha)$.

%Про вариант $(\alpha\rightarrow\alpha)\rightarrow(\alpha\rightarrow\alpha)$ мы уже разобрались, теперь
%рассмотрим $\alpha\rightarrow\alpha$.

%Это может быть либо $f$, либо $a S^{\alpha\rightarrow\alpha}$, либо $b S^{\alpha\rightarrow\alpha}$,
%либо их комбинация: 
%$\lambda x:\alpha.S_1^{\alpha\rightarrow\alpha}(S_2^{\alpha\rightarrow\alpha}\dots S_n^{\alpha\rightarrow\alpha}(z)))$,


\end{proof}