\begin{theorem} О слабой нормализации. Пусть некоторый лямбда-терм имеет
тип $\sigma$. Тогда существует конечная последовательность бета-редукций,
приводящая к нормальной форме.
\end{theorem}

%\begin{lemma}
%Пусть $A$ и $B$ --- лямбда-термы, а $x$ --- некоторая переменная.
%Тогда $h(A[x := B]) \le max(h(A),h(B),)$.
%\end{lemma}
%
%\begin{proof}
%\end{proof}

\begin{proof}Доказательство теоремы о слабой нормализации.

Мы покажем утверждение теоремы, доказав, что последовательность редукций,
при которой редуцируется самый вложенный редекс максимальной степени
(такой редекс $R = (\lambda x:\sigma.A^\tau) B$ выражения 
$E = \dots R \dots$, что $h(R) = h(E)$, и $h(A) < h(R)$, как и $h(B) < h(R)$),
приводит к нормальной форме.

Рассмотрим, что произойдет с количеством редексов высоты $h(R)$ в результате редукции 
$R$. Рассмотрим некоторый редекс $P$ в новом выражении $F = \dots A[x := B] \dots$, 
получающемся из $E$ путем редукции $R$. Он может:
\begin{itemize}
\item находиться целиком вне результата редукции редекса $R$, тогда он никак не изменяется при 
данной редукции;
\item являться правой частью внешнего редекса $F = \dots (\lambda y:\theta.Q) A[x:=B] \dots$.
В этом случае, поскольку тип переменной $x$ совпадает с типом
терма $B$, изменения высоты данного редекса не произойдет;
\item являться левой частью внешнего редекса $F = \dots A^\theta[x:=B] Q^\rho \dots$.
В этом случае неизбежно $A = \lambda y:\rho.C^\theta$ для некоторого $C$, 
и $h(A[x:=B] Q) = h(\rho\rightarrow\theta) < h((\lambda x:\sigma.\lambda y:\rho.C) B Q) =
h(\sigma\rightarrow\rho\rightarrow\theta)$, то есть, в этом случае появившийся редекс
будет иметь высоту, меньшую $h(R)$;
\item находиться целиком внутри результата редукции. Заметим, что из $R: \sigma \rightarrow \tau$ 
следует, что $h(R) = h(\sigma\rightarrow\tau) > h(\sigma)$, и по принципу построения 
последовательности редукций, $h(R) > h(A)$ и $h(R) > h(B)$.
Тогда, воспользовавшись леммой, заключаем $h(A[x:=B]) \le max(h(A),h(B),h(\sigma)) < h(R)$.
\end{itemize}
Таким образом, поскольку сам редекс $R$ будет разрушен, и никаких новых редексов данной или 
большей высоты не добавится, количество редексов высоты $h(R)$ уменьшится минимум на 1.

Рассмотрим функцию $m(E) = (h(E), n_E(h(E)))$, где $n_E(x)$ --- количество
редексов высоты $x$ в формуле $E$. Если $n_E(h(E)) > 1$, то редукция самого вложенного редекса 
степени $h(E)$ уменьшит $n_E(h(E))$ минимум на 1, если же их остался один --- то
редукция устранит их совсем, не добавив новых. 
Значит, мы получим строго убывающую последовательность $m(E)$, ограниченную снизу: она
прервется, когда в выражении не останется ни одного редекса. То есть, данная последовательность
редукций приведет выражение к нормальной форме.

%\end{enumerate}
\end{proof}